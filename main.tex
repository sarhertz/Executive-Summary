\documentclass{article}
\usepackage[utf8]{inputenc}

\title{MPP Proposal}
\author{sarhertz }
\date{February 2018}

\begin{document}


Executive Summary of George Orwell’s “Politics and the English Language”				
In his essay, “Politics and the English Language”, George Orwell argues that contemporary writing is vague and unimaginative. He demonstrates that writers are lazy in their choice of diction, and that they complicate language unnecessarily. Political writers in particular, he claims, use stale metaphors and long words to avoid uncomfortable topics. The following is a summary of Orwell’s suggestions on how to improve your writing:

Do not rely upon figures of speech (i.e. metaphors and similes) which you hear or read all the time. They no longer mean anything and hardly anyone uses them properly.					 				
The shorter the better. Avoid words with too many syllables, especially Latinate or Greek words which intimidate and confuse people.	 	
Use verbs and nouns properly, and make sure the relationship between the subject and object is clear in your sentences. Never use the passive voice.
Avoid obscure, disciplinary jargon. Consider whether your words have an agreed-upon definition, or whether they are broad, catch-all terms like ‘democracy’.
Cut, cut, cut. Only keep that which is absolutely necessary.					 						
Above all, Orwell advocates that writers carefully consider words and their meanings. In sharing university governance decisions with the general public, brevity and accessibility are key. Long and intimidating documents will only distort the message, communicating nothing or something else entirely.


\maketitle

\section{Introduction}

\end{document}
